
%%%%%%%%%%%%%%%%%%%%%%%%%%%%%%%%%%%%%%%%%%%%%%%%%%%%%%%%%%%%%%%%%%%%
%% I, the copyright holder of this work, release this work into the
%% public domain. This applies worldwide. In some countries this may
%% not be legally possible; if so: I grant anyone the right to use
%% this work for any purpose, without any conditions, unless such
%% conditions are required by law.
%%%%%%%%%%%%%%%%%%%%%%%%%%%%%%%%%%%%%%%%%%%%%%%%%%%%%%%%%%%%%%%%%%%%

\documentclass[
  digital, %% This option enables the default options for the
           %% digital version of a document. Replace with `printed`
           %% to enable the default options for the printed version
           %% of a document.
  table,   %% Causes the coloring of tables. Replace with `notable`
           %% to restore plain tables.
  nolof,     %% Prints the List of Figures. Replace with lof/`nolof` to
           %% hide the List of Figures.
  nolot,     %% Prints the List of Tables. Replace with lot/`nolot` to
           %% hide the List of Tables.
  %% More options are listed in the user guide at
  %% <http://mirrors.ctan.org/macros/latex/contrib/fithesis/guide/mu/fi.pdf>.
	draft, %% ked bude napisane, tak zmenit na final
]{fithesis3}
%% The following section sets up the locales used in the thesis.
\usepackage[resetfonts]{cmap} %% We need to load the T2A font encoding
\usepackage[T1,T2A]{fontenc}  %% to use the Cyrillic fonts with Russian texts.
\usepackage[
  main=english, %% By using `czech` or `slovak` as the main locale
                %% instead of `english`, you can typeset the thesis
                %% in either Czech or Slovak, respectively.
  german, russian, czech, slovak %% The additional keys allow
]{babel}        %% foreign texts to be typeset as follows:
%%
%%   \begin{otherlanguage}{german}  ... \end{otherlanguage}
%%   \begin{otherlanguage}{russian} ... \end{otherlanguage}
%%   \begin{otherlanguage}{czech}   ... \end{otherlanguage}
%%   \begin{otherlanguage}{slovak}  ... \end{otherlanguage}
%%
%% For non-Latin scripts, it may be necessary to load additional
%% fonts:
\usepackage{paratype}
\def\textrussian#1{{\usefont{T2A}{PTSerif-TLF}{m}{rm}#1}}
%%
%% The following section sets up the metadata of the thesis.
\thesissetup{
    date          = \the\year/\the\month/\the\day,
    university    = mu,
    faculty       = fi,
    type          = mgr,
    author        = Adrian,
    gender        = m,
    advisor       = John Smith,
    title         = {Thesis title},
    TeXtitle      = {Thesis title},
    keywords      = {keyword1, keyword2, ...},
    TeXkeywords   = {keyword1, keyword2, \ldots},
    abstract      = {This is the abstract of my thesis, which can

                     span multiple paragraphs.},
    thanks        = {This is the acknowledgement for my thesis, which can

                     span multiple paragraphs.},
    bib           = bibliothesis.bib,
}
\usepackage{makeidx}      %% The `makeidx` package contains
\makeindex                %% helper commands for index typesetting.
%% These additional packages are used within the document:
\usepackage{paralist} %% Compact list environments
\usepackage{amsmath}  %% Mathematics
\usepackage{amsthm}
\usepackage{amsfonts}
\usepackage{url}      %% Hyperlinks
\usepackage{markdown} %% Lightweight markup
\usepackage{listings} %% Source code highlighting
\lstset{
  basicstyle      = \ttfamily,%
  identifierstyle = \color{black},%
  keywordstyle    = \color{blue},%
  keywordstyle    = {[2]\color{cyan}},%
  keywordstyle    = {[3]\color{olive}},%
  stringstyle     = \color{teal},%
  commentstyle    = \itshape\color{magenta}}
%%begin, pridane
\usepackage{hyperref}
\hypersetup{
    colorlinks,
    citecolor=black,
    filecolor=black,
    linkcolor=black,
    urlcolor=black
}

\thesisload
%%end
\begin{document}
\chapter{Introduction}
Intro

\chapter{Chapter}
\section{Definition}
\subsection{Univariate polynomial}
A univariate polynomial is a mathematical expression of the form
\begin{align}
       a_{n}x^{n}  +  a_{n-1}x^{n-1} +  \ldots  +  a_{1}x  &+  a_{0} \label{eq:polynom}
\end{align}
where the $a_{n}, a_{n-1}, \ldots, a_{1}, a_{0}$  are the coefficients of the polynomial and $x$ is called an indeterminate or a variable.  The highest $n \geq 0$ is called the degree of the polynomial (such $n$ exists, since the set $\{i \, | \, f_{i} \neq 0 \}$ is finite) and $a_{n} \neq 0$ is called the leading coefficient \parencite{rosicky07}. 

\subsection{Roots of a univariate polynomial}
Let R be a ring, $f = a_{n}x^{n}  +  a_{n-1}x^{n-1} +  \ldots  +  a_{1}x  +  a_{0}$ a polynomial of $R[x]$, $c \in R$. Then an element $a_{n}c^{n}  +  a_{n-1}c^{n-1} +  \ldots  +  a_{1}c  +  a_{0}$ is called a value of the polynomial and we denote it as $f(c)$.\\
Using this, we can create a polynomial function by mapping every element of $R$, $x$, to the result of a substitution $f(x) = a_{n}x^{n}  +  a_{n-1}x^{n-1} +  \ldots  +  a_{1}x  +  a_{0}$ \parencite{polynomialsChina}.

Let $f$ be a polynomial over $R$, $c \in R$. We say that $c$ is a root of the polynomial $f$ if $f(c) = 0$ \parencite{rosicky07}.


\printbibliography[heading=bibintoc]


  \makeatletter\thesis@blocks@clear\makeatother
  \phantomsection %% Print the index and insert it into the
  \addcontentsline{toc}{chapter}{\indexname} %% table of contents.
  \printindex

\appendix %% Start the appendices.
\chapter{An appendix}
Here you can insert the appendices of your thesis.

\end{document}
