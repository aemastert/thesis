
%%%%%%%%%%%%%%%%%%%%%%%%%%%%%%%%%%%%%%%%%%%%%%%%%%%%%%%%%%%%%%%%%%%%
%% I, the copyright holder of this work, release this work into the
%% public domain. This applies worldwide. In some countries this may
%% not be legally possible; if so: I grant anyone the right to use
%% this work for any purpose, without any conditions, unless such
%% conditions are required by law.
%%%%%%%%%%%%%%%%%%%%%%%%%%%%%%%%%%%%%%%%%%%%%%%%%%%%%%%%%%%%%%%%%%%%

\documentclass[
  digital, %% This option enables the default options for the
           %% digital version of a document. Replace with `printed`
           %% to enable the default options for the printed version
           %% of a document.
  table,   %% Causes the coloring of tables. Replace with `notable`
           %% to restore plain tables.
  nolof,     %% Prints the List of Figures. Replace with lof/`nolof` to
           %% hide the List of Figures.
  nolot,     %% Prints the List of Tables. Replace with lot/`nolot` to
           %% hide the List of Tables.
  %% More options are listed in the user guide at
  %% <http://mirrors.ctan.org/macros/latex/contrib/fithesis/guide/mu/fi.pdf>.
	draft, %% ked bude napisane, tak zmenit na final
]{fithesis3}
%% The following section sets up the locales used in the thesis.
\usepackage[resetfonts]{cmap} %% We need to load the T2A font encoding
\usepackage[T1,T2A]{fontenc}  %% to use the Cyrillic fonts with Russian texts.
\usepackage[
  main=english, %% By using `czech` or `slovak` as the main locale
                %% instead of `english`, you can typeset the thesis
                %% in either Czech or Slovak, respectively.
  german, russian, czech, slovak %% The additional keys allow
]{babel}        %% foreign texts to be typeset as follows:
%%
%%   \begin{otherlanguage}{german}  ... \end{otherlanguage}
%%   \begin{otherlanguage}{russian} ... \end{otherlanguage}
%%   \begin{otherlanguage}{czech}   ... \end{otherlanguage}
%%   \begin{otherlanguage}{slovak}  ... \end{otherlanguage}
%%
%% For non-Latin scripts, it may be necessary to load additional
%% fonts:
\usepackage{paratype}
\def\textrussian#1{{\usefont{T2A}{PTSerif-TLF}{m}{rm}#1}}
%%
%% The following section sets up the metadata of the thesis.
\thesissetup{
    date          = \the\year/\the\month/\the\day,
    university    = mu,
    faculty       = fi,
    type          = mgr,
    author        = Adrian,
    gender        = m,
    advisor       = John Smith,
    title         = {Thesis title},
    TeXtitle      = {Thesis title},
    keywords      = {keyword1, keyword2, ...},
    TeXkeywords   = {keyword1, keyword2, \ldots},
    abstract      = {This is the abstract of my thesis, which can

                     span multiple paragraphs.},
    thanks        = {This is the acknowledgement for my thesis, which can

                     span multiple paragraphs.},
    bib           = bibliothesis.bib,
}
\usepackage{makeidx}      %% The `makeidx` package contains
\makeindex                %% helper commands for index typesetting.
%% These additional packages are used within the document:
\usepackage{paralist} %% Compact list environments
\usepackage{amsmath}  %% Mathematics
\usepackage{amsthm}
\usepackage{amsfonts}
\usepackage{url}      %% Hyperlinks
\usepackage{markdown} %% Lightweight markup
\usepackage{algorithm}
\usepackage[noend]{algpseudocode}
\usepackage{listings} %% Source code highlighting
\lstset{
  basicstyle      = \ttfamily,%
  identifierstyle = \color{black},%
  keywordstyle    = \color{blue},%
  keywordstyle    = {[2]\color{cyan}},%
  keywordstyle    = {[3]\color{olive}},%
  stringstyle     = \color{teal},%
  commentstyle    = \itshape\color{magenta}}
%%begin, pridane
\usepackage{hyperref}
\hypersetup{
    colorlinks,
    citecolor=black,
    filecolor=black,
    linkcolor=black,
    urlcolor=black
}

\thesisload
%%end
\begin{document}
\chapter{Introduction}
Intro

\chapter{Chapter}
\section{Definitions}
\subsection{Univariate polynomial}
A univariate polynomial $f$ over a ring $R$ is a mathematical expression of the form
\begin{align}
       f = a_{n}x^{n}  +  a_{n-1}x^{n-1} +  \ldots  +  a_{1}x  &+  a_{0} \label{eq:polynom}
\end{align}
where the $a_{n}, a_{n-1}, \ldots, a_{1}, a_{0}$  are the coefficients of the polynomial and elements of $R$, and $x$ is called an indeterminate or a variable.  The highest $n \geq 0$ is called the degree of the polynomial (such $n$ exists, since the set $\{i \, | \, f_{i} \neq 0 \}$ is finite) and $a_{n} \neq 0$ is called the leading coefficient \parencite{rosicky07}. 

\subsection{Roots of a univariate polynomial}
Let R be a ring, $f = a_{n}x^{n}  +  a_{n-1}x^{n-1} +  \ldots  +  a_{1}x  +  a_{0}$ a polynomial of $R[x]$, $c \in R$. Then an element $a_{n}c^{n}  +  a_{n-1}c^{n-1} +  \ldots  +  a_{1}c  +  a_{0}$ is called a value of the polynomial and we denote it as $f(c)$.

Using this, we can create a polynomial function by mapping every element $x$ of $R$, to the result of a substitution $f(x) = a_{n}x^{n}  +  a_{n-1}x^{n-1} +  \ldots  +  a_{1}x  +  a_{0}$ \parencite{polynomialsChina}.

Let $f$ be a polynomial over $R$, $c \in R$. We say that $c$ is a root of the polynomial $f$ if $f(c) = 0$ \parencite{rosicky07}.

\section{Approximation of a root using iterative methods}
The iterative methods generally require knowledge of one or more initial guesses for the desired root(s) of the polynomial. This often poses a problem itself and there are techniques and methods for finding them. The simplest method for finding a guess is by looking at the plot of the polynomial, which is often not possible (e.g. when dealing with very complex and long polynomials). Some of these methods will be shown later and thus for this section, we will assume we already have a guess.

\subsection{Bisection method}
The simplest method for finding a better approximation of a root is bisection method \parencite{rootApproxMeth}. Assume that function $f(x)$ is continuous on interval $[a,b]$ and that $f(a)f(a) < 0$. Then according to the intermediate value theorem \parencite{interValue} there must be at least one root in $[a,b]$. The interval may be chosen large enough that there is more than one root, this is not a problem however, since the algorithm will always converge to some root $\alpha$ in $[a,b]$ and a smaller interval containing only one root. Since all polynomial functions are continuous \parencite{polyCont}, we can use this theorem to create an algorithm.

\newcommand*\Let[2]{\State #1 $\gets$ #2}
\algrenewcommand\algorithmicrequire{\textbf{Precondition:}}
\algrenewcommand\algorithmicensure{\textbf{Postcondition:}}
\begin{algorithm}
  \caption{Bisection algorithm
    \label{alg:bisect}}
  \begin{algorithmic}[1]
    \Require{$f$ polynomial function, $a$, $b$ interval bounds, $\epsilon$ precision error}
    \Statex
    \Function{Bisection}{$f, a, b, \epsilon$}
      \Let{$c$}{$\frac{a + b}{2}$}
      \If{$c - a \leq \epsilon$}
				\Return{$c$}
			\EndIf
      \If{$f(a) * f(c) < 0$}
				\State \Return{\Call{Bisection}{$f, a, c, \epsilon$}}
				\Else \State \Return{\Call{Bisection}{$f, c, b, \epsilon$}}
			\EndIf
    \EndFunction
  \end{algorithmic}
\end{algorithm}

\theoremstyle{definition}
\newtheorem{example}{Example}[section]
\begin{example}
Find the largest root $\rho$ of 
\begin{align}
      2x^{4} - 3x - 2
\end{align}

with the precision $\epsilon = 0.00005$.
\end{example}

It is pretty easy to see that the largest root is located between $1 < \rho < 2$, so we will use this interval as our initial guess. The results are shown in the table below.

\begin{table}
  \begin{tabularx}{\textwidth}{lll}
    \toprule
    Iteration & $c_{n}$ & $f(c_{n})$\\
    \midrule
    1 & 1.500000 & 8.89063 \\
    2 & 1.250000 & 1.56470 \\
    3 & 1.125000 & -0.09771 \\
    4 & 1.187500 & 0.61665 \\
    5 & 1.156250 & 0.23327 \\
    6 & 1.140625 & 0.06158 \\
    7 & 1.132813 & -0.01957 \\
    8 & 1.136719 & 0.02062 \\
    9 & 1.134766 & 0.00043 \\
    10 & 1.133790 & -0.00959 \\
    11 & 1.134278 & -0.00458 \\
    12 & 1.134522 & -0.00208 \\
    13 & 1.134644 & -0.00082 \\
    14 & 1.134705 & -0.00020 \\
    15 & 1.134736 & 0.00012 \\
    \bottomrule
  \end{tabularx}
  \caption{Bisection algorithm on example todo:numbering of examples}
  \label{tab:bis}
\end{table}


\printbibliography[heading=bibintoc]


  \makeatletter\thesis@blocks@clear\makeatother
  \phantomsection %% Print the index and insert it into the
  \addcontentsline{toc}{chapter}{\indexname} %% table of contents.
  \printindex

\appendix %% Start the appendices.
\chapter{An appendix}
Here you can insert the appendices of your thesis.

\end{document}
